\chapter*{Wstęp}
\addcontentsline{toc}{chapter}{Wstęp}
\section{Powód powstania projektu}
W dzisiejszym świecie ilość danych tekstowych rośnie z niesamowitą prędkością. Dane tekstowe stają się nieodłączną częścią wielu obszarów, w tym mediów społecznościowych, forów, blogów i innych platform, gdzie użytkownicy aktywnie dzielą się swoimi opiniami i informacjami. Jedną z najpopularniejszych platform jest YouTube, na której miliony użytkowników codziennie zostawiają komentarze pod filmami. Analiza tych danych tekstowych może dostarczyć cenne informacje do badań marketingowych, społecznych oraz do poprawy interakcji z publicznością.

\section{Idea projektu}
Ten projekt to aplikacja GUI przeznaczona do analizy danych tekstowych. Dane tekstowe mogą pochodzić z dowolnych dokumentów tekstowych. Projekt skupia się na przetwarzaniu komentarzy z YouTube, co umożliwia użytkownikom łatwe i szybkie pobieranie oraz analizowanie tych danych. Po wybraniu określonych metod przetwarzania komentarzy, które użytkownik może dostosować, można ocenić wynik pracy jako 2D wykres, na którym dane tekstowe zostaną wizualnie podzielone na grupy (klastry). Aby pomóc w jakościowej analizie danych, użytkownik będzie mógł dostosować metody przetwarzania języka naturalnego (NLP), wektoryzacji i klasteryzacji.

\section{Dla kogo to}
W ten sposób projekt dostarcza potężne narzędzie do jakościowej analizy danych tekstowych, które może być przydatne zarówno badaczom, jak i specjalistom z dziedziny marketingu i mediów społecznościowych.

