\chapter*{Wstęp}
\addcontentsline{toc}{chapter}{Wstęp}
\section{Powód realizacji projektu}
		W dzisiejszym świecie ilość danych tekstowych rośnie w niewiarygodnym tempie. Dane tekstowe stają się integralną częścią wielu obszarów, w tym mediów społecznościowych, forów, blogów i innych platform, na których użytkownicy aktywnie dzielą się swoimi opiniami i informacjami. Jedną z najpopularniejszych platform jest YouTube, gdzie miliony użytkowników codziennie zostawiają komentarze pod filmami. Analiza tych danych tekstowych może dostarczyć cennych informacji do badań marketingowych, badań społecznych i poprawy zaangażowania odbiorców.

\section{Pomysł na projekt}
 		Ten projekt jest aplikacją GUI przeznaczoną do analizy danych tekstowych. Dane tekstowe mogą być dowolnymi dokumentami tekstowymi. Projekt koncentruje się na przetwarzaniu komentarzy z YouTube, co pozwala użytkownikom na łatwe i szybkie pobieranie i analizowanie tych danych. Po wybraniu określonych metod przetwarzania komentarzy, które mogą być dostosowane przez użytkownika, użytkownik może ocenić wynik pracy w postaci końcowego wykresu 2D, na którym dane tekstowe zostaną wizualnie podzielone na grupy (klastry). Metody NLP, wektoryzacji i klastrowania będą dostępne dla użytkownika, aby pomóc w analizie danych jakościowych. 
 	

 \section{Dla kogo to jest}
 		W ten sposób projekt zapewnia potężne narzędzie do jakościowej analizy danych tekstowych, które może być przydatne zarówno dla badaczy, jak i specjalistów ds. marketingu i mediów społecznościowych.

