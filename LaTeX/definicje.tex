\documentclass[10pt, a4paper]{book}
\pagestyle{plain}
\usepackage[a4paper,left=3cm,right=3cm,top=3cm,bottom=3cm]{geometry}
\geometry{a4paper}

\usepackage[utf8]{inputenc}
\usepackage{amsmath}
\usepackage{amssymb}
\usepackage{amsthm}
\usepackage{amsbsy}
\usepackage{latexsym}
\usepackage{indentfirst}
\usepackage[onehalfspacing]{setspace}
\usepackage{marvosym}
\usepackage{listings}
\lstset{ %
  language=Pascal,                % określamy język programowania dla kodu 
  basicstyle=\tiny,               % wielkość czcionki dla kodu programu
  showstringspaces=false,         % podkreślanie spacji wewnątrz łańcuchów
  numbers=left,                   % gdzie umieszczać numery linii
  numberstyle=\footnotesize,      % wielkość czcionki dla numerów linii
  stepnumber=1,                   % krok między dwiema  numerami linii. Jeśli  1 to  każda linia będzie numerowana
  numbersep=5pt,                  % jak daleko numery linii są odsunięte od kodu 
  showspaces=false,               % pokaż spacje w obrębie łańcuchów za pomocą podkreślników
  showtabs=false,                 % pokaż tabulacje w obrębie łańcuhów za pomocą podkreślników
  frame=single,                   % dodaj obramowanie wokół kodu
  tabsize=2,                      % ustaw domyślny rozmiar tabulacji równy dwóm spacjom
  captionpos=b,                   % ustaw pozycję dla podpisu - podpis na dole 
  breaklines=true,                % włącz automatyczne łamanie linii
  breakatwhitespace=false,        % ustaw jeżeli automatyczne łamanie ma następować tylko na białych znakach
  %keywordstyle=\ttfamily\color{green}, % jakim kolorem i jaką czcionką będą wyświetlane słowa kluczowe
  %identifierstyle=\ttfamily\color{yellow}\bfseries, %jakim kolorem i jaką czcionką będą wyświetlane identyfikatory
  %commentstyle=\color{brown}, %jakim kolorem i jaką czcionką będą wyświetlane komentarze
  stringstyle=\ttfamily, %jakim kolorem i jaką czcionką będą wyświetlane łańcuchy znakowe
  escapeinside={\%*}{*)},          % jeżeli chcesz umieścić komentarze w obrębie swojego kodu,
  inputencoding=utf8, % Pakiet listings nie obsługuje polskich znaków UTF-8, ale można to "naprawić" - polskie znaki będą zastępowane komendami LaTeX - patrz niżej
  literate={ą}{{\k{a}}}1 
  {Ą}{{\k{A}}}1
  {ę}{{\k{e}}}1
  {Ę}{{\k{E}}}1
  {ó}{{\'o}}1
  {Ó}{{\'O}}1
  {ś}{{\'s}}1
  {Ś}{{\'S}}1
  {ł}{{\l{}}}1
  {Ł}{{\L{}}}1
  {ż}{{\.z}}1
  {Ż}{{\.Z}}1
  {ź}{{\'z}}1
  {Ź}{{\'Z}}1
  {ć}{{\'c}}1
  {Ć}{{\'C}}1
  {ń}{{\'n}}1
  {Ń}{{\'N}}1
}
\usepackage{fancyvrb}
%\usepackage{clrscode}
\usepackage{amsmath, amsthm, amssymb}
\usepackage{amsfonts}
\usepackage[polish]{babel}
\usepackage[OT4]{fontenc}
\usepackage{graphicx}
\usepackage{epstopdf}
\usepackage{subfig}
\usepackage{wrapfig}
\usepackage{longtable}
\usepackage{verbatim}
\usepackage{float}
\usepackage{psfrag}
\usepackage{makecell}
\usepackage{wrapfig}
\usepackage{array}
\linespread{1}

\DeclareGraphicsRule{.tif}{png}{.png}{`convert #1 `dirname #1`/`basename #1 .tif`.png}

\clubpenalty=3000
\widowpenalty=600
\hyphenpenalty=8000
\sloppy

\newcommand{\FullFileListing}[1]{\VerbatimInput[frame=lines,label={[Plik #1]Koniec pliku #1},framesep=1em,numbers=left,numbersep=2pt,fontsize=\footnotesize]{../application/#1}}
\DefineVerbatimEnvironment{pyline}{Verbatim}{frame=lines,fontsize=\footnotesize}
\DefineVerbatimEnvironment{pylinex}{Verbatim}{frame=leftline,fontsize=\footnotesize}
\DefineVerbatimEnvironment{pylinen}{Verbatim}{frame=lines,fontsize=\footnotesize,numbers=left}
\newcommand{\zlamLinie}{\Rewind{}}

\newcounter{defscounter}
\setcounter{tocdepth}{2}

\theoremstyle{definition}
\newtheorem{odesolution}[defscounter]{Definicja}
\newtheorem{odeexample}{Przykład}

\newtheorem{hyperbolicPDE}[defscounter]{Definicja}

\renewcommand{\lstlistlistingname}{Spis listingów}
\renewcommand{\lstlistingname}{Listing}

\newenvironment{changemargin}[1]{%
\begin{list}{}{%
\setlength{\topsep}{0pt}%
\setlength{\leftmargin}{#1}%
\setlength{\rightmargin}{\parindent}%
\setlength{\listparindent}{\parindent}%
\setlength{\itemindent}{\parindent}%
\setlength{\parsep}{\parskip}%
}%
\item[]}{\end{list}}

\makeatletter
 \renewcommand\@seccntformat[1]{\csname the#1\endcsname. }
 \renewcommand\numberline[1]{#1.\hskip0.7em}
\makeatother

\usepackage{tabularx}
\usepackage{color, colortbl}
\definecolor{Gray}{gray}{0.9}
