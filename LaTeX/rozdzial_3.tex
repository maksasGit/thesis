\chapter{Rozdzial 3}

\section{Opis aplikacji}
	\subsection{Okno pracy}
		Graficzna aplikacja składa się z okna z paskiem menu zawierającym dwa pola ("File" i "Comments"), oraz główną strefą, w której znajduje się menu zakładek. Każda zakładka odpowiada osobnej przestrzeni roboczej (workspace). Przyjrzyjmy się tym elementom bliżej.

		
	\subsection{Menu}
		Pasek menu umożliwia szybką nawigację po oknie pracy w dowolnym momencie. W menu zaimplementowane są następujące funkcje:
		\begin{itemize}
			\item Otwieranie okna dialogowego w celu utworzenia nowej przestrzeni roboczej ("File" -> "New Project").
			\item Otwieranie okna dialogowego w celu pobrania komentarzy z YouTube ("Comments" -> "YouTube").
		\end{itemize}

		Podglądamy bardziej szczegółowo sposób korzystania z tych funkcji:

		
		\subsubsection{Tworzenie przestrzeni roboczej}
			Aby utworzyć nową przestrzeń roboczą, należy podać nazwę przestrzeni roboczej (Workspace name) i plik źródłowy (Source file) zawierający dane tekstowe. Początkowe okno \textbf{CreateWorkspaceWindowEmpty}. Użytkownik musi podać dowolną nazwę dla przestrzeni roboczej i wybrać ścieżkę do pliku tekstowego z danymi. Można to zrobić ręcznie lub wybrać plik za pomocą okna dialogowego FileDialog z biblioteki PyQt, co pozwala odnaleźć potrzebny plik na komputerze. Przykład wypełnionego poprawnie okna \textbf{CreateWorkspaceWindowFill}.

		
		\subsubsection{Pobieranie komentarzy}
			Do pobrania komentarzy służy okno dialogowe, w którym należy podać link do interesującego nas filmu, jak pokazano na zrzucie ekranu \textbf{CommentsWindowStep1}. Jeśli film zostanie znaleziony, w polu Video Info pojawią się informacje o nim, pozwalając użytkownikowi upewnić się, że to właśnie ten film, komentarze z którego chcemy pobrać. Następnie użytkownik musi kliknąć przycisk Download comments, który zapisze wszystkie komentarze.
		
	\subsection{Workspace}
		Po utworzeniu przynajmniej jednej przestrzeni roboczej pojawia się menu nawigacyjne zakładek, gdzie każda zakładka odpowiada osobnej przestrzeni roboczej. Można swobodnie przełączać się między tymi zakładkami i zamykać je. Zawartość każdej zakładki obejmuje:
		\begin{itemize}
			\item Pole na wykresy.
			\item Przyciski Previous i Next do przełączania wykresów w polu na wykresy.
			\item Przycisk Text do otwierania okna dialogowego zarządzania przetwarzaniem tekstu.
			\item Przycisk Vectorization do otwierania okna dialogowego zarządzania wektoryzacją.
			\item Przycisk Cluster do otwierania okna dialogowego zarządzania klasteryzacją.
			\item Przycisk Apply do zastosowania wszystkich wybranych parametrów i otwarcia okna z wynikami klasteryzacji.
		\end{itemize}
		
	\subsubsection{Pole na wykresy}
			Pole na wykresy w każdej chwili może wyświetlać jeden z trzech proponowanych wykresów:
			\begin{itemize}
				\item Wykres częstości słów (freq). 
				\item Wykres długości tekstu (length).
				\item Wykres najbliższych sąsiadów (k-nearest). 
			\end{itemize}					
			
			Za pomocą przycisków Previous i Next można przełączać wykresy odpowiednio do przodu i do tyłu.		
					
		
	\subsection{Text}
			Przycisk Text otwiera okno dialogowe, w którym użytkownik może określić parametry przetwarzania tekstu. Oto jak domyślne okno wygląda \textbf{TextProcessingWindowStandard}. Domyślnie wszystkie parametry są ustawione na wartość true, co oznacza, że są używane.

			Użytkownik może wybrać jedną z proponowanych opcji przetwarzania tekstu:

			\begin{itemize}
				\item Alpha -- podstawowa obróbka opisana wcześniej
				\item Stop-words -- usuwanie stop słów opisane wcześniej
				\item Lematize -- lematyzacja opisana wcześniej
			\end{itemize}
			
	\subsection{Wektoryzacja}
		Przycisk Vectorization otwiera okno dialogowe, w którym użytkownik może wybrać jeden z proponowanych metod wektoryzacji i określić parametry dla wybranej metody. Oto jak domyślne okno wygląda \textbf{VectorizationWindowStandard}. Domyślnie wybrana jest metoda TF-IDF z domyślnymi parametrami.
		Następnie wymieniono listę dostępnych metod i parametrów dla każdej metody:
		\begin{itemize}
			\item Metoda TF-IDF. Dostępne parametry do konfiguracji:
			\begin{itemize}
				\item max\_df: Zakres dozwolonych wartości ()
				\item min\_df: Zakres dozwolonych wartości ()
			\end{itemize}
			\item Metody Word2vec. Dostępne parametry do konfiguracji:
			\begin{itemize}
				\item vector\_size: Zakres dozwolonych wartości ()
				\item window: Zakres dozwolonych wartości ()
				\item min\_count: Zakres dozwolonych wartości ()
				\item sg: Zakres dozwolonych wartości ()
			\end{itemize}
		\end{itemize}
			
	\subsection{Cluster}
		Przycisk Cluster otwiera okno dialogowe, w którym użytkownik może wybrać jedną z proponowanych metod klasteryzacji i określić parametry dla wybranej metody. Oto jak domyślne okno wygląda \textbf{ClusterWindowStandard}. Domyślnie wybrana jest metoda K-means z domyślnymi parametrami.
		Następnie wymieniono listę dostępnych metod i parametrów dla każdej metody:
		\begin{itemize}
			\item Metoda K-means. Dostępne parametry do konfiguracji:
			\begin{itemize}
				\item n\_cluster: Zakres dozwolonych wartości ()
				\item max\_iter: Zakres dozwolonych wartości ()
				\item n\_iter: Zakres dozwolonych wartości ()
			\end{itemize}
			\item Metody DBSCAN. Dostępne parametry do konfiguracji:
			\begin{itemize}
				\item min\_samples: Zakres dozwolonych wartości ()
				\item eps: Zakres dozwolonych wartości ()
			\end{itemize}
		\end{itemize}
			
		\subsection{Result Window}		
			Przycisk Apply otwiera nowe okno z wynikami wykonanej pracy. Okno to zawiera pole z wykresem (2D plot) i pasek narzędziowy (toolbar), gdzie użytkownik może skorzystać z gotowych funkcji zaproponowanych przez bibliotekę Matplotlib do nawigacji po wykresie i innych operacji. 
			

\section{Przykład użycia}



	\underline{how use image}
	\begin{figure}
		\centering
		\scalebox{0.25} {\includegraphics[width=0.5\textwidth]{./img/rys1.jpg	}}
		\caption{Przyklad rys}
		\label{fig:rys1}
	\end{figure}
		
	\subsection{Pobieranie danych tekstowych}

	\subsection{Tworzenie Workspace}

	\subsection{Wybór metod i parametrów} 

		
	\subsection{Przetwarzanie tekstu} 


		
		\subsubsection{Alpha}

				
				
		\subsubsection{Stop-words}

			
		\subsubsection{Lemmatize}


			
	\subsection{Wektoryzacja}

	
		\subsubsection{TF-IDF}

			
		\subsubsection{Word2Vec}

	
	
	\subsection{Klasteryzacja}

		\subsubsection{K-means}

		
		\subsubsection{DBSCAN}

	
	\subsection{Analiza wynkiow pracy}
		
		
		

