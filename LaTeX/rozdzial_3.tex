\chapter{Rozdzial 3}

\section{Opis aplikacji}
	\subsection{Okno pracy}
		Graficzna aplikacja składa się z okna z paskiem menu zawierającym dwa pola ("File" i "Comments"), oraz główną strefą, w której znajduje się menu zakładek. Każda zakładka odpowiada osobnej przestrzeni roboczej (workspace). Przyjrzyjmy się tym elementom bliżej.

		
	\subsection{Menu}
		Pasek menu umożliwia szybką nawigację po oknie pracy w dowolnym momencie. W menu zaimplementowane są następujące funkcje:
		\begin{itemize}
			\item Otwieranie okna dialogowego w celu utworzenia nowej przestrzeni roboczej ("File" -> "New Project").
			\item Otwieranie okna dialogowego w celu pobrania komentarzy z YouTube ("Comments" -> "YouTube").
		\end{itemize}

		Podglądamy bardziej szczegółowo sposób korzystania z tych funkcji:

		
		\subsubsection{Tworzenie przestrzeni roboczej}
			Aby utworzyć nową przestrzeń roboczą, należy podać nazwę przestrzeni roboczej (Workspace name) i plik źródłowy (Source file) zawierający dane tekstowe. Oto jak wygląda puste okno \textbf{CreateWorkspaceWindowEmpty}. Użytkownik musi podać dowolną nazwę dla przestrzeni roboczej i wybrać ścieżkę do pliku tekstowego z danymi. Można to zrobić ręcznie lub wybrać plik za pomocą okna dialogowego FileDialog z biblioteki PyQt, co pozwala odnaleźć potrzebny plik na komputerze. Przykład wypełnionego poprawnie okna \textbf{CreateWorkspaceWindowFill}.

		
		\subsubsection{Pobieranie komentarzy}
			Do pobrania komentarzy służy okno dialogowe, w którym należy podać link do interesującego nas filmu, jak pokazano na zrzucie ekranu \textbf{CommentsWindowStep1}. Jeśli film zostanie znaleziony, w polu Video Info pojawią się informacje o nim, pozwalając użytkownikowi upewnić się, że to właśnie ten film, komentarze z którego chcemy pobrać. Następnie użytkownik musi kliknąć przycisk Download comments, który zapisze wszystkie komentarze.
		
	\subsection{Workspace}
		Po utworzeniu przynajmniej jednej przestrzeni roboczej pojawia się menu nawigacyjne zakładek, gdzie każda zakładka odpowiada osobnej przestrzeni roboczej. Można swobodnie przełączać się między tymi zakładkami i zamykać je. Zawartość każdej zakładki obejmuje:
		\begin{itemize}
			\item Pole na wykresy.
			\item Przyciski Previous i Next do przełączania wykresów w polu na wykresy.
			\item Przycisk Text do otwierania okna dialogowego zarządzania przetwarzaniem tekstu.
			\item Przycisk Vectorization do otwierania okna dialogowego zarządzania wektoryzacją.
			\item Przycisk Cluster do otwierania okna dialogowego zarządzania klasteryzacją.
			\item Przycisk Apply do zastosowania wszystkich wybranych parametrów i otwarcia okna z wynikami klasteryzacji.
		\end{itemize}
		
	\subsubsection{Pole na wykresy}
			Pole na wykresy w każdej chwili może wyświetlać jeden z trzech proponowanych wykresów:
			\begin{itemize}
				\item Wykres częstości słów (freq). 
				\item Wykres długości tekstu (length).
				\item Wykres najbliższych sąsiadów (k-nearest). 
			\end{itemize}					
			
			Za pomocą przycisków Previous i Next można przełączać wykresy odpowiednio do przodu i do tyłu.		
					
		
	\subsection{Text}
			Przycisk Text otwiera okno dialogowe, w którym użytkownik może określić parametry przetwarzania tekstu. Oto jak domyślne okno wygląda \textbf{TextProcessingWindowStandard}. Domyślnie wszystkie parametry są ustawione na wartość true, co oznacza, że są używane.

			Użytkownik może wybrać jedną z proponowanych opcji przetwarzania tekstu:

			\begin{itemize}
				\item Alpha - podstawowa obróbka opisana wcześniej
				\item Stop-words - usuwanie stop słów opisane wcześniej
				\item Lematize - lematyzacja opisana wcześniej
			\end{itemize}
			
	\subsection{Wektoryzacja}
		Przycisk Vectorization otwiera okno dialogowe, w którym użytkownik może wybrać jeden z proponowanych metod wektoryzacji i określić parametry dla wybranej metody. Oto jak domyślne okno wygląda \textbf{VectorizationWindowStandard}. Domyślnie wybrana jest metoda TF-IDF z domyślnymi parametrami.
		Następnie wymieniono listę dostępnych metod i parametrów dla każdej metody:
		\begin{itemize}
			\item Metoda TF-IDF. Dostępne parametry do konfiguracji:
			\begin{itemize}
				\item max\_df: Zakres dozwolonych wartości ()
				\item min\_df: Zakres dozwolonych wartości ()
			\end{itemize}
			\item Metody Word2vec. Dostępne parametry do konfiguracji:
			\begin{itemize}
				\item vector\_size: Zakres dozwolonych wartości ()
				\item window: Zakres dozwolonych wartości ()
				\item min\_count: Zakres dozwolonych wartości ()
				\item sg: Zakres dozwolonych wartości ()
			\end{itemize}
		\end{itemize}
			
	\subsection{Cluster}
		Przycisk Cluster otwiera okno dialogowe, w którym użytkownik może wybrać jedną z proponowanych metod klasteryzacji i określić parametry dla wybranej metody. Oto jak domyślne okno wygląda \textbf{ClusterWindowStandard}. Domyślnie wybrana jest metoda K-means z domyślnymi parametrami.
		Następnie wymieniono listę dostępnych metod i parametrów dla każdej metody:
		\begin{itemize}
			\item Metoda K-means. Dostępne parametry do konfiguracji:
			\begin{itemize}
				\item n\_cluster: Zakres dozwolonych wartości ()
				\item max\_iter: Zakres dozwolonych wartości ()
				\item n\_iter: Zakres dozwolonych wartości ()
			\end{itemize}
			\item Metody DBSCAN. Dostępne parametry do konfiguracji:
			\begin{itemize}
				\item min\_samples: Zakres dozwolonych wartości ()
				\item eps: Zakres dozwolonych wartości ()
			\end{itemize}
		\end{itemize}
			
		\subsection{Result Window}		
			Przycisk Apply otwiera nowe okno z wynikami wykonanej pracy. Okno to zawiera pole z wykresem (2D plot) i pasek narzędziowy (toolbar), gdzie użytkownik może skorzystać z gotowych funkcji zaproponowanych przez bibliotekę Matplotlib do nawigacji po wykresie i innych operacji. 
			

\section{Пример использования}


	\underline{how use image}
	\begin{figure}
		\centering
		\scalebox{0.25} {\includegraphics[width=0.5\textwidth]{./img/rys1.jpg	}}
		\caption{Przyklad rys}
		\label{fig:rys1}
	\end{figure}
		
	\subsection{Получение текстовых данных}
		
	\subsection{Создание Workspace}
		
	\subsection{Выбор методов и параметров} 
		Возможных камбинаций методов и парметров может быть большое количество количество и достаточно сложно предугадать итоговый результат работы. Подробно рассмотрим как каждый элемент влияет влияет на тестовых данных:
		
	\subsection{Текстовая обработка} 
		Далее будет расмотренно использование всех трех методов, то есть активен при обаботке будет только один. Оцениновать эфиктивность этих методов можно с помощью графкиков (freq, lenght). Сравнивая графики исходных данных и графики после конкретного метода обработки.
		
		\subsubsection{Alpha}
			Исходные графики, графики после. Визуальны видны изменения в длинее общий длине комментариев, все они стали немного меньше, более это заметно на графике freq где мы можем увидеть теперь более встречаемые слова. 
				
				
		\subsubsection{stop-words}
			Исходные графики, графики после. Визуально графики недолжны были сильно измениться ведь шум от alpha не дает это сделать. Однако в комбинации с alpha обработкой можено заметить что слова которые остовались только после alpha обработки исчезли так как ("the" , "in", "a") считаются наиболее встрчеающимися словами в анлглийском языке. Однако stop-words включает в себя эти слова.
			
		\subsubsection{lemmatize}
			При использовании только лемматизации сложно заметить какое-то весомое влияние этой функции.
			Более заметно оно станет только при использовании всех трех методоов. Когда особо встречаемыае смогут обьединиться так можно заметить на примере (human , humans) обьединились в одну группу.
			
	\subsection{Векторизация}
		При дальнейшем расмотрении работы этих методов, мы будем расматривать их отдельно. Рассмотрим как работает каждый из этих метдоов на необработаных текстовых данны и обработаных, а так же рассмотрим влияние изменений параметров. Для анализа наших результатов нам потребуются полностью выполнить весь процес анализа данных, чтобы визуализировать наши векторезированные данные, так как необходмо выбрать метод кластеризации будет использоваться метод K-means с 1 кластером.
	
		\subsubsection{TF-IDF}
			При исполльзование на необработаных данных
			При исполльзование на обработаных данных
			Изменяем параметры min , max
			Замечаем типичные результаты с центром и то что рядом стоящие точки никак не связаны.
			
		\subsubsection{Word2Vec}
		 	При исполльзование на необработаных данных
		 	При исполльзование на обработаных данных
		 	Изменяем параметры vector\_size , window, min\_count, sg
		 	Замечаем типичные результаты ( при увелечения вектора как буд-то все растягивается и становиться дальше). 
	
	
	\subsection{Кластеризация}
		Далее расмотрим методы кластеризации, расмотрим их одтельно. Расмотрим их на обработаных данных с зафиксироваными параметрами векторизации word2vec. 
		
		\subsubsection{k-means}
			Рассмотрим как работает увеличение количество кластеров. И остальных
		
		\subsubsection{DBSCAN}
			Рассмотрим как работает увелечение eps, min\_pts и попробуем подобрать значения основываясь на графике k-nears
			
			Заметим что сложно подобрать эти значения.
	
	\subsection{Интерпритиация результатов = Итог}
		
		
		

