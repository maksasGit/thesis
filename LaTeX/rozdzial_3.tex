\chapter{Rozdzial 3}

\section{Описание работы}
	\subsection{Рабочее окно}
		Графическое приложение представляет собой окно с меню-баром, содержащим два поля ("File" и "Comments"), и основной зоной, где располагается таб-меню. Каждый таб представляет собой отдельное рабочее пространство (workspace). Рассмотрим эти элементы детальнее.
		
	\subsection{Меню}
		Меню-бар позволяет быстро навигировать по рабочему окну в любой момент. В меню-баре реализованы следующие функции:
		\begin{itemize}
			\item Открытие диалогового окна для создания нового рабочего пространства ("File" -> "New Project").
			\item Открытие диалогового окна для получения комментариев с YouTube ("Comments" -> "YouTube").
		\end{itemize}

		Рассмотрим более подробно использование этих функций:
		
		\subsubsection{Создавние workspace}
			Для создания нового рабочего пространства необходимо указать имя рабочей области (Workspace name) и файл-источник (Source file) с текстовыми данными. Вот как выглядит пустое окно \textbf{CreateWorkspaceWindowEmpty}. Пользователю необходимо ввести любое имя для рабочего пространства, а также указать путь к текстовому файлу с данными. Это можно сделать вручную или выбрать файл с помощью диалогового окна FileDialog из библиотеки PyQt, что позволяет найти необходимый файл на компьютере. Пример правильно заполненного окна \textbf{CreateWorkspaceWindowFill}.
		
		\subsubsection{Получение комментариев}
			Для получения комментариев используется диалоговое окно, в котором необходимо указать ссылку на интересующее видео, как показано на скриншоте \textbf{CommentsWindowStep1}. Если видео найдено, то в поле Video Info появится информация о нем, позволяющая пользователю убедиться, что это именно то видео, комментарии которого необходимо получить. Далее пользователю остается только нажать на кнопку Download comments, которая сохранит все комментарии.
				
	\subsection{Workspace}
		После создания хотя бы одного рабочего пространства появляется меню навигации по вкладкам, где каждая вкладка соответствует отдельному рабочему пространству. По этим вкладкам можно свободно переключаться и закрывать их. Содержимое каждой вкладки включает:
		\begin{itemize}
			\item Поле для графиков.
			\item Кнопки Previous и Next для переключения графиков в поле для графиков.
			\item Кнопку Text для открытия диалогового окна управления обработкой текста.
			\item Кнопку Vectorization для открытия диалогового окна управления векторизацией.
			\item Кнопку Cluster для открытия диалогового окна управления кластеризацией.
			\item Кнопку Apply для применения всех выбранных параметров и открытия окна с результатами кластеризации.
		\end{itemize}
		
	\subsubsection{Поле для графиков}
			Поле для графиков в каждый момент времени может отображать один из трех предложенных графиков:
			\begin{itemize}
				\item График частот слов (freq). Описан ранее в 
				\item График длин текста(length). Описан ранее в
				\item График ближайших соседей (k-nearest). Описан ранее в
			\end{itemize}					
			
			С помощью кнопок Previous и Next графики можно переключать соответственно вперед и назад.		
		
	\subsection{Text}
			Кнопка Text открывает диалоговое окно, в котором пользователь может указать параметры для обработки текста. Вот как выглядит окно по умолчанию \textbf{TextProcessingWindowStandard}. По умолчанию все параметры установлены на значение true, то есть используются.
		Пользователь может выбрать один из предложеных опций обработки текста:
		\begin{itemize}
			\item Alpha - примитиваная обработка описанная ранее
			\item Stop-words - удаленее стоп слов описаная ранее
			\item Lematize - лемматизациия описаная ранее
		\end{itemize}
			
		\subsubsection{Vectorization}
			Кнопка Vectorization открывает диалоговое окно, в котором пользователь может выбрать один из предложенных методов векторизации и указать параметры для выбранного метода. Вот как выглядит окно по умолчанию \textbf{VectorizationWindowStandard}. По умолчанию выбран метод TF-IDF с параметрами по умолчанию.
			Далее описан список доступных методов и параметров для каждого метода:
			\begin{itemize}
				\item Метод TF-IDF. Доступные параметры для настройки:
					\begin{itemize}
						\item 1
						\item 2
					\end{itemize}
				\item Методы Word2vec. Доступные парметры для настройки:
					\begin{itemize}
						\item 1
						\item 2
						\item 3
						\item 4
					\end{itemize}
			\end{itemize}
			
		\subsubsection{Cluster}
			Кнопка Cluster открывает диалоговое окно, в котором пользователь может выбрать один из предложенных методов кластеризации и указать параметры для выбранного метода. Вот как выглядит окно по умолчанию \textbf{ClusterWindowStandard}. По умолчанию выбран метод K-means с параметрами по умолчанию.
			Далее описан список доступных методов и параметров для каждого метода:
			\begin{itemize}
				\item Метод K-means. Доступные параметры для настройки:
					\begin{itemize}
						\item 1
						\item 2
					\end{itemize}
				\item Методы DBSCAN. Доступные парметры для настройки:
					\begin{itemize}
						\item 1
						\item 2
					\end{itemize}
			\end{itemize}
			
		\subsubsection{Result Window}		
			Кнопка Apply открывает новое окно с результатами проделанной работы. Данное окно включает поле с графиком (2D plot) и панель инструментов (toolbar), где пользователь может воспользоваться готовыми функциями, предложенными библиотекой Matplotlib, для навигации по графику и других операций. Окно сохраняется, и пользователь может анализировать результаты, закрывать окно или сворачивать его и продолжать работу. Для упрощения навигации по всем окнам с результатами (ResultWindows) они будут иметь уникальные названия, чтобы можно было легко идентифицировать нужное окно.
			

\section{Пример использования}
		
	\subsection{Получение комментариев}	
	\subsection{Создание Workspace}
	\subsection{Выбор методов и параметров}
	\subsection{Интерпритиация результатов}

