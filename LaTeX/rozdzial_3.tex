\chapter{Rozdzial 3}

\section{Описание приложения}
	\subsection{Рабочее окно}
		Графическое приложение представляет собой окно с меню-баром, содержащим два поля ("File" и "Comments"), и основной зоной, где располагается таб-меню. Каждый таб представляет собой отдельное рабочее пространство (workspace). Рассмотрим эти элементы детальнее.
		
	\subsection{Меню}
		Меню-бар позволяет быстро навигировать по рабочему окну в любой момент. В меню-баре реализованы следующие функции:
		\begin{itemize}
			\item Открытие диалогового окна для создания нового рабочего пространства ("File" -> "New Project").
			\item Открытие диалогового окна для получения комментариев с YouTube ("Comments" -> "YouTube").
		\end{itemize}

		Рассмотрим более подробно использование этих функций:
		
		\subsubsection{Создавние workspace}
			Для создания нового рабочего пространства необходимо указать имя рабочей области (Workspace name) и файл-источник (Source file) с текстовыми данными. Вот как выглядит пустое окно \textbf{CreateWorkspaceWindowEmpty}. Пользователю необходимо ввести любое имя для рабочего пространства, а также указать путь к текстовому файлу с данными. Это можно сделать вручную или выбрать файл с помощью диалогового окна FileDialog из библиотеки PyQt, что позволяет найти необходимый файл на компьютере. Пример правильно заполненного окна \textbf{CreateWorkspaceWindowFill}.
		
		\subsubsection{Получение комментариев}
			Для получения комментариев используется диалоговое окно, в котором необходимо указать ссылку на интересующее видео, как показано на скриншоте \textbf{CommentsWindowStep1}. Если видео найдено, то в поле Video Info появится информация о нем, позволяющая пользователю убедиться, что это именно то видео, комментарии которого необходимо получить. Далее пользователю остается только нажать на кнопку Download comments, которая сохранит все комментарии.
				
	\subsection{Workspace}
		После создания хотя бы одного рабочего пространства появляется меню навигации по вкладкам, где каждая вкладка соответствует отдельному рабочему пространству. По этим вкладкам можно свободно переключаться и закрывать их. Содержимое каждой вкладки включает:
		\begin{itemize}
			\item Поле для графиков.
			\item Кнопки Previous и Next для переключения графиков в поле для графиков.
			\item Кнопку Text для открытия диалогового окна управления обработкой текста.
			\item Кнопку Vectorization для открытия диалогового окна управления векторизацией.
			\item Кнопку Cluster для открытия диалогового окна управления кластеризацией.
			\item Кнопку Apply для применения всех выбранных параметров и открытия окна с результатами кластеризации.
		\end{itemize}
		
	\subsubsection{Поле для графиков}
			Поле для графиков в каждый момент времени может отображать один из трех предложенных графиков:
			\begin{itemize}
				\item График частот слов (freq). Описан ранее в 
				\item График длин текста(length). Описан ранее в
				\item График ближайших соседей (k-nearest). Описан ранее в
			\end{itemize}					
			
			С помощью кнопок Previous и Next графики можно переключать соответственно вперед и назад.		
		
	\subsection{Text}
			Кнопка Text открывает диалоговое окно, в котором пользователь может указать параметры для обработки текста. Вот как выглядит окно по умолчанию \textbf{TextProcessingWindowStandard}. По умолчанию все параметры установлены на значение true, то есть используются.
		Пользователь может выбрать один из предложеных опций обработки текста:
		\begin{itemize}
			\item Alpha - примитиваная обработка описанная ранее
			\item Stop-words - удаленее стоп слов описаная ранее
			\item Lematize - лемматизациия описаная ранее
		\end{itemize}
			
		\subsubsection{Vectorization}
			Кнопка Vectorization открывает диалоговое окно, в котором пользователь может выбрать один из предложенных методов векторизации и указать параметры для выбранного метода. Вот как выглядит окно по умолчанию \textbf{VectorizationWindowStandard}. По умолчанию выбран метод TF-IDF с параметрами по умолчанию.
			Далее описан список доступных методов и параметров для каждого метода:
			\begin{itemize}
				\item Метод TF-IDF. Доступные параметры для настройки:
					\begin{itemize}
						\item max\_df: Диапозон допустимых значений ()
						\item min\_df: Диапозон допустимых значений ()
					\end{itemize}
				\item Методы Word2vec. Доступные парметры для настройки:
					\begin{itemize}
						\item vector\_size: Диапозон допустимых значений ()
						\item window: Диапозон допустимых значений ()
						\item min\_count: Диапозон допустимых значений ()
						\item sg: Диапозон допустимых значений ()
					\end{itemize}
			\end{itemize}
			
		\subsubsection{Cluster}
			Кнопка Cluster открывает диалоговое окно, в котором пользователь может выбрать один из предложенных методов кластеризации и указать параметры для выбранного метода. Вот как выглядит окно по умолчанию \textbf{ClusterWindowStandard}. По умолчанию выбран метод K-means с параметрами по умолчанию.
			Далее описан список доступных методов и параметров для каждого метода:
			\begin{itemize}
				\item Метод K-means. Доступные параметры для настройки:
					\begin{itemize}
						\item n\_cluster: Диапозон допустимых значений ()
						\item max\_iter: Диапозон допустимых значений ()
						\item n\_iter: Диапозон допустимых значений ()
					\end{itemize}
				\item Методы DBSCAN. Доступные парметры для настройки:
					\begin{itemize}
						\item min\_samples: Диапозон допустимых значений ()
						\item eps: Диапозон допустимых значений ()
					\end{itemize}
			\end{itemize}
			
		\subsubsection{Result Window}		
			Кнопка Apply открывает новое окно с результатами проделанной работы. Данное окно включает поле с графиком (2D plot) и панель инструментов (toolbar), где пользователь может воспользоваться готовыми функциями, предложенными библиотекой Matplotlib, для навигации по графику и других операций. Окно сохраняется, и пользователь может анализировать результаты, закрывать окно или сворачивать его и продолжать работу. Для упрощения навигации по всем окнам с результатами (ResultWindows) они будут иметь уникальные названия, чтобы можно было легко идентифицировать нужное окно.
			

\section{Пример использования}
	\begin{figure}
		\centering
		\scalebox{0.25} {\includegraphics[width=0.5\textwidth]{./img/rys1.jpg	}}
		\caption{Przyklad rys}
		\label{fig:rys1}
	\end{figure}
		
	\subsection{Получение текстовых данных}
		
	\subsection{Создание Workspace}
		
	\subsection{Выбор методов и параметров} 
		Возможных камбинаций методов и парметров может быть большое количество количество и достаточно сложно предугадать итоговый результат работы. Подробно рассмотрим как каждый элемент влияет влияет на тестовых данных:
		
	\subsection{Текстовая обработка} 
		Далее будет расмотренно использование всех трех методов, то есть активен при обаботке будет только один. Оцениновать эфиктивность этих методов можно с помощью графкиков (freq, lenght). Сравнивая графики исходных данных и графики после конкретного метода обработки.
		
		\subsubsection{Alpha}
			Исходные графики, графики после. Визуальны видны изменения в длинее общий длине комментариев, все они стали немного меньше, более это заметно на графике freq где мы можем увидеть теперь более встречаемые слова. 
				
				
		\subsubsection{stop-words}
			Исходные графики, графики после. Визуально графики недолжны были сильно измениться ведь шум от alpha не дает это сделать. Однако в комбинации с alpha обработкой можено заметить что слова которые остовались только после alpha обработки исчезли так как ("the" , "in", "a") считаются наиболее встрчеающимися словами в анлглийском языке. Однако stop-words включает в себя эти слова.
			
		\subsubsection{lemmatize}
			При использовании только лемматизации сложно заметить какое-то весомое влияние этой функции.
			Более заметно оно станет только при использовании всех трех методоов. Когда особо встречаемыае смогут обьединиться так можно заметить на примере (human , humans) обьединились в одну группу.
			
	\subsection{Векторизация}
		При дальнейшем расмотрении работы этих методов, мы будем расматривать их отдельно. Рассмотрим как работает каждый из этих метдоов на необработаных текстовых данны и обработаных, а так же рассмотрим влияние изменений параметров. Для анализа наших результатов нам потребуются полностью выполнить весь процес анализа данных, чтобы визуализировать наши векторезированные данные, так как необходмо выбрать метод кластеризации будет использоваться метод K-means с 1 кластером.
	
		\subsubsection{TF-IDF}
			При исполльзование на необработаных данных
			При исполльзование на обработаных данных
			Изменяем параметры min , max
			Замечаем типичные результаты с центром и то что рядом стоящие точки никак не связаны.
			
		\subsubsection{Word2Vec}
		 	При исполльзование на необработаных данных
		 	При исполльзование на обработаных данных
		 	Изменяем параметры vector\_size , window, min\_count, sg
		 	Замечаем типичные результаты ( при увелечения вектора как буд-то все растягивается и становиться дальше). 
	
	
	\subsection{Кластеризация}
		Далее расмотрим методы кластеризации, расмотрим их одтельно. Расмотрим их на обработаных данных с зафиксироваными параметрами векторизации word2vec. 
		
		\subsubsection{k-means}
			Рассмотрим как работает увеличение количество кластеров. И остальных
		
		\subsubsection{DBSCAN}
			Рассмотрим как работает увелечение eps, min\_pts и попробуем подобрать значения основываясь на графике k-nears
			
			Заметим что сложно подобрать эти значения.
	
	\subsection{Интерпритиация результатов = Итог}
		
		
		

